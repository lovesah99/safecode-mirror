%%-----------------------------------------------------------------------------
\section{Run-time Checks and Instrumentation}
\label{section:checks}
%%-----------------------------------------------------------------------------

%%-----------------------------------------------------------------------------
\subsection{Complete vs. Incomplete Checks}
\label{section:checks:complete}
%%-----------------------------------------------------------------------------

Most of the SAFECode run-time checks come in two flavors: complete
and incomplete.  A complete check attempts to find the bounds of a
memory object into which a pointer points; if it cannot find the
memory object, then the pointer must be invalid, and the check fails.

The problem with complete checks is that they only work when the
compiler knows everything that can be known about a memory object.
Sadly, this isn't always true; applications are linked with native
code libraries compiled with other compilers (unthinkable, I know, but
it happens); the SAFECode compiler cannot analyze native code, and so
it does not know about memory objects allocated or freed by the
library code (also known as \emph{external code}).

Incomplete checks are SAFECode's way of permitting a mixture of
SAFECode-compiled code and native external code; if a pointer could be
manipulated by external code, SAFECode relaxes its run-time checks so
that failure to find the referant memory object does not cause a
run-time check to fail.

By default, SAFECode makes all of its checks incomplete checks (this
is because each compilation unit treats other compilation units as
external code).  When used with libLTO, SAFECode can use points-to
analysis to determine which incomplete checks can be converted into
complete checks without causing false positives.

Incomplete checks admit the possibility that memory safety errors will escape
detection.  However, they make memory safety usable in practice, and
so we use them.

%%-----------------------------------------------------------------------------
\subsection{Run-time Checks}
\label{section:checks:checks}
%%-----------------------------------------------------------------------------

Below are the run-time checks that SAFECode may add to a program.
Note that many of these functions have alternate versions for pointers
that are determined to be incomplete or unknown by SAFECode's
points-to analysis algorithm.

\begin{itemize}
\item{\tt poolcheck (void * pool, void * ptr, size\_t length)}: \\
The {\tt poolcheck} call is used to instrument loads and stores to
memory (including LLVM atomic operations).  It ensures that the
pointer points within a memory object in the pool and that the load or
store will not read/write past the end of the memory object.

\item{\tt fastlscheck (void * ptr, void * start, size\_t objsz,
size\_t len)}: \\
The {\tt fastlscheck()} function is identical to the {\tt poolcheck()}
function in functionality; the difference is that {\tt fastlscheck()}
is passed the bounds of the memory object into which the pointer
should point.  It is an optimized version of {\tt poolcheck()} that
does not need to search for object bounds information in a side data
structure.

\item{\tt poolcheck\_align (void * pool, void * ptr)}: \\
The {\tt poolcheck\_align()} function is used when type-safe
load/store optimizations are enabled.  It is possible for a pointer which
is type-safe to be loaded from a memory object which is not type-safe.
When a type-safe pointer is loaded via a type-inconsistent pointer,
{\tt poolcheck\_align()} verifies that the loaded pointer points
within the specified pool at the correctly aligned offset for objects
of its type.  This ensures that no further checks are needed when the
type-safe poiner is used for loads and stores.

\item{\tt free\_check (void * pool, void * ptr)}: \\
The {\tt free\_check()} function checks that the pointer points to
the beginning of a valid heap object.  It is used to catch invalid
{\tt free} calls for allocators not known to tolerate invalid
deallocation requests.

\item{\tt boundscheck (void * pool, void * src, void * dest)}: \\
The {\tt boundscheck()} function takes a source pointer and a
destination pointer that is computed from the source pointer; the
checks first determines whether the source pointer is within a valid
memory object within the specified pool and, if so, that the
destination pointer is within the same memory object.  It is primarily
used for performing array and structure indexing checks on LLVM {\tt
getelementptr} instructions.

If the destination pointer goes out of bounds, then {\tt
boundscheck()} returns a \emph{rewrite pointer}.  A rewrite pointer
(or \emph{OOB pointer}) point to an unmapped portion of the address
space.  They are used to allow pointers to go out of bounds so long as
they are not dereferenced.

\item{\tt exactcheck (void * src, void * dest, void * base, int
objsize)}: \\
The {\tt exactcheck()} function is a fast version of the {\tt
boundscheck()} function that does not need to do an object bounds
lookup.

\item{\tt funccheck (void * pool, void f()}: \\
The {\tt funccheck()} function determines if a function pointer
belongs to the set of valid function pointer targets for an indirect
function call.  It is used to ensure control-flow integrity.
\end{itemize}

SAFECode also instruments code with other functions to support the
above run-time checks:

\begin{itemize}
\item{\tt getActualValue()}:
The {\tt getActualValue()} function takes a value and determines if it
is a rewrite pointer.  If it is, it returns the actual out-of-bounds
value that the rewrite pointer represents.  Otherwise, it returns the
original value.

The {\tt getActualValue()} function is primarily used for supporting
the comparison of pointers that have gone outside their object bounds.

\item{\tt pool\_register()}:
The {\tt pool\_register()} family of functions register the bounds of
allocated memory objects in side data-structures; these are used to
map a pointer to the memory object to which it belongs in run-time
checks.

Note that some memory objects may not be registered if SAFECode determines
that their bounds are never needed.

\item{\tt pool\_reregister()}:
The {\tt pool\_reregister()} function unregisters a memory object and
registers a new object of the specified size.  It is designed to
support allocators like {\tt realloc()}.
\end{itemize}
